\begin{definition}
    Rust uses an \textbf{ownership model} to manage memory, with three fundamental rules.
    \begin{enumerate}
    \item Every value has an owner (variable/struct).
    \item Every owner is unique.
    \item When the owner goes out of scope, the value is dropped (freed).
    \end{enumerate}
    \end{definition}
    
    \begin{lstlisting}
    
    fn main() {
        // allocated on the heap
        let string = "Hello".to_string();
        // dropped
    } // 'string' goes out of scope, so "Hello" freed
    
    
    \end{lstlisting}
    
    
    \begin{lstlisting}
    fn say_string(string: String) {
        println!("{}", string);
    }
    
    fn main() {
        let string = "Hello".to_string();
        say_string(string);
        // running say_string(string) again would give an error, as the value
        // is used after the move.
    } 
    
    
    \end{lstlisting}
    
    Ownership is checked at compile time. One approach would be to clone the string with string.clone and to then call the function on it. \\
    
    Note that this ownership model \underline{does not} impact ``cheap types" like integers, booleans, and floats.
    
    \subsection{References}
    We can use `\&' for references. For example, we could pass a reference to an initial string and call $say_string$ on it just fine.
    
    \begin{lstlisting}
    fn say_string(string: &String) {
        println!("{}", string);
    }
    
    fn main() {
        let string = "Hello".to_string();
        say_string(&string);
        say_string(&string);
    }
    \end{lstlisting} 
        
    Another example:
    \begin{lstlisting}
    fn main() { // this is okay, as a loses ownership to b.
        let b;
        {
            let a = "Hello".to_string();
            say_string(&a);
            b = a;
        }
        say_string(&b);
    }
    \end{lstlisting}
    
    Let's now try to implement a counter.
    
    \begin{lstlisting}
    struct Counter {
        count: u32,
    }
    
    impl Counter {
        fn get_count(&self) -> u32 {
            self.count
        }
    
        fn increment(&mut_self) {
            self.count += 1
        }
    }
    
    fn main() {
        let mut counter = Counter {count : 0}; 
        // mutable reference needs to have mutable ``var'' owner
    }
    \end{lstlisting}
    
    % Add box around this
    \textbf{You can take one mutable reference or as many immutable references as you want at a time (but not both).}
    
    \begin{lstlisting}
        let mut vec: vec![1, 2, 3];
        let first = &vec[0];
        vec.clear(); // mutable reference
    
        printn!("{}", first) // first is an immutable borrow, and we cannot have both at the same time
    \end{lstlisting}
    \end{document}
    
    
    
    
    
    