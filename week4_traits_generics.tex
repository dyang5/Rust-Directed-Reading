\section{Traits and Generics}

\subsection{Generics} Generics are used to create definitions for items like function signatures or structs, which we can then use with many different concrete data types.

\begin{lstlisting}[frame = none]
pub enum Either {
    Left(String),
    Right(i32),
}

impl Either {
    pub fn into_left(self) -> Option<String> {
        match self {
            Either::Left(string) => Some(string),
            Either::Right(_) => None,
        }
    }
}
\end{lstlisting}

To make ``Either" more generic, we can use \textbf{type parameters} on the Either type to make it generic over any two types. \\

\begin{lstlisting}[frame = none]
pub enum Either<L, R> {
    Left(L),
    Right(R),
}
\end{lstlisting}
\begin{definition}
Here, we refer to \codeword{L} and \codeword{R} are generic types.
\end{definition}

We can now do declarations as follows:
\begin{lstlisting}[frame = none]
// The `L` and `R` types are replaced with `String` and `i32` respectively for these.
let message: Either<String, i32> = Either::Left("Hello, world!".to_string());
let integer: Either<String, i32> = Either::Right(5);
\end{lstlisting}

Note that the impl blocks need to be changed as well:
\begin{lstlisting}[frame = none]
impl<L, R> Either<L, R> {
    //  ^^^^^^ We need this now
        pub fn into_left(self) -> Option<L> { // <- was `Option<String>` before
            match self {
                Either::Left(left) => Some(left),
                Either::Right(_right) => None,
            }
        }
    
        // other methods here
    } 
// We can call `.replace_left(...)` with any type we want, here's with `f32`
let float_or_int: Either<f32, i32> = string_or_int.replace_left(5.0);
\end{lstlisting}

Here's the structure for a function that will swap the left and right sides:
\begin{lstlisting}[frame = none]
fn swap<L, R>(val: Either<L, R>) -> Either<R, L> {
    match val {
        Either::Left(left: L) -> Either::Right(left),
        Either::Right(right: R) -> Either::Left(right),
    }
}
\end{lstlisting}

\subsection{Traits}
\begin{definition}
Traits define shared behavior between different types.
\end{definition}

Let's give a general framework for where we can use traits:
\begin{lstlisting}[]
// define tweets and books
struct Tweet {
    username: String,
    content: String,
    likes: u32,
}

struct Book {
    author: String,
    title: String,
    content: String,
}

// use traits instead (for similar structures)

trait Summary {
    fn summarize(&self) -> String;
}    

// Implementing functions to summarize tweets and books
impl Summary for Tweet {
    fn summarize(&self) -> String {
        format!("@{}: {}", self.username, self.content)
    }
}

impl Summary for Book {
    fn summarize(&self) -> String {
        format!("{}, by {}", self.title, self.author)
    }
}
\end{lstlisting}


The key of the above code is the \codeword{trait} type for summaries!

\begin{lstlisting}
trait Summary {
    fn summarize(&self) -> String;
}    
\end{lstlisting}

\begin{definition}
We can use trait bounds to abstract traits where you have a function or struct/enum with a generic type that must implement some set of traits.
\end{definition}

\begin{lstlisting}[frame = none]
// Takes a generic `T`, but _only_ if the T type implements `Summary`!
fn describe<T: Summary>(text: T) {
    // `text` can do anything that `Summary` can do because of the trait bound
    println!("Here's a short summary of the text: {}", text.summarize());
}

let tweet = Tweet {
    username: "swarthmore".to_string(),
    content: "Only 12 more days until spring semester classes begin! We can't wait to welcome our students back to campus.".to_string(),
    likes: 35,
};

let book = Book {
    author: "Mara Bos".to_string(),
    title: "Atomics and Locks".to_string(),
    content: "-- omitted -".to_string(),
};

describe(tweet);
describe(book);
    
\end{lstlisting}

We can also use \textbf{Blanket Traits}, which can be done as follows.

\begin{lstlisting}[frame = none]
trait Print {
    fn print(&self);
}

// implementing Print for anything that has Display
impl<T: std:fmt::Display> Print for T {
    fn print(&self) {
        println!("{}", self);
    }
}
\end{lstlisting}

The three main uses for traits include
\begin{itemize}
    \item Interfaces
    \item Operator Overloading
    \item Type Markers
\end{itemize} 

Summary: 
\begin{itemize}
    \item Generics allow for defining types and functions that work on values of different types.
    \item Traits are like interfaces that types can implement.
    \item Types and functions can use trait bounds on generic types to restrict which types can be used.
    \item Monomorphization is the process where the compiler looks at every usage of a generic and turns it into its own copy of the function or type.
    \item Monomorphization can lead to more optimization, but slower compile times in some extreme cases.
    \item Traits allow for operator overloading, shared interfaces, and type markers for the compiler like Copy.
\end{itemize}
