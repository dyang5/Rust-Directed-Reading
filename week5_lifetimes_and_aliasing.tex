\section{Lifetimes and Aliasing}

\begin{definition}[Lifetimes]
The lifetime of something is the duration of a program's execution during which it exists
\end{definition}

The following fails in C:
\begin{lstlisting}[language = C, frame = none]
int* foo() {
  int x = 3;
  return &x;
}
\end{lstlisting}

due to lifetimes, as the lifetime of $x$ is restricted to the function \codeword{foo()}. \\

We can write the following in Rust to address lifetimes:
\begin{lstlisting}[frame = none]
fn get_default<'a>(map: &'a HashMap<String, String>, key: &'a String, default: &'a String) -> &'a String {
  match map.get(key) {
      Some(val) => val,
      None => default,
  }
}
\end{lstlisting}

Here, \codeword{'a} is a generic \textit{lifetime} parameter.
The return value lives at least as long as whichever lives the shortest between map, key, and default. \\

However, the following code 
\begin{lstlisting}[frame = none]
let default = "DEFAULT".to_string();
let mut map = HashMap::new();
map.insert("Hello".to_string(), "World".to_string());
let value;
{
    let key = "missing".to_string(); // `key` allocated here
    value = get_default(&map, &key, &default);
} // `key` dropped here
// `map`, `default`, and `value` still valid
println!("{}", value);
  
\end{lstlisting}
causes problems due to the lifetime of key. Instead, we can use the following code, which will compile.

\begin{lstlisting}[frame = none]
fn get_default<'a>(map: &'a HashMap<String, String>, key: &String, default: &'a String) -> &'a String {
  match map.get(key) {
      Some(val) => val,
      None => default,
  }
}
\end{lstlisting}

Something can ``expire" in terms of its lifetime if it is moved out of scope, or if it is borrowed elsewhere. \\

References in structs \textbf{need to have lifetimes}. We can introduce the following Salad example:
\begin{lstlisting}[frame = none]
struct Salad<'a> {
  lettuce: &'a str,
  dressing: &'a str,
}

impl<'a> Salad <'a> {
  fn new(lettuce: &'a str, dressing: &'a str) -> Salad<'a> {
    Salad {
      lettuce,
      dressing,
    }
  }
}

fn main() {
  let lettuce = "lettuce".to_string();
  let dressing = "ranch".to_string();
  let salad = Salad::new(&lettuce, &dressing);
}
\end{lstlisting}

works! But the following does not (due to lifetimes):
\begin{lstlisting}[frame = none]
fn main() {
  let dressing = "ranch".to_string();
  let salad = {
    let lettuce = "lettuce".to_string();
    let salad = Salad::new(&lettuce, &dressing);
    salad
  } // lettuce lifetime is in block
}
\end{lstlisting}

\begin{definition}
A static reference is one that lives forever (e.g. string literals such as lettuce and dressing in the examples above).
\end{definition}
