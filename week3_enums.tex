\section{Enums} 
\begin{definition}
    In Rust, enums can be used to define a type that can take on one of multiple possible variants.
\end{definition}

\begin{lstlisting}[frame = none]
enum Shape {
    Circle,
    Rectangle,
}

let what_shape: String = match circle {
    Shape::Circle => "circle".to_string(),
    _ => "not a circle".to_string(), // _ is a "wildcard" type
};

fn main() {
    let circle = Shape::Circle;

    match circle {
        Shape::Circle => println!("Circle"),
        Shape::Rectangle => println!("Rectangle"),
    }
}
\end{lstlisting}

We can also add data to each variant, either by making it a tuple variant or a struct variant. \\

\begin{definition}Tuple variants have unnamed fields, while struct variants have named fields.
\end{definition}

\begin{lstlisting}[frame = none]
enum Shape {
    Circle(f64),
    Rectangle {
        width: f64,
        height: f64,
    }
}

fn main() {
    let circle = Shape::Circle(5.0);

    match circle {
        Shape::Circle(radius) => {
            println!("Circle with radius {radius}");
        }
        Shape::Rectangle { width: w, height: h } => {
            println!("Rectangle that is {w} by {h}");
        }
    }

}
\end{lstlisting}

Note that enums are also types, so we can pass in different variants of a given structure to a function expecting that structure:
\begin{lstlisting}[frame = none]
impl Shape {
    fn area(&self) -> f64 {
        match self {
            Shape::Rectangle { width, height } => {
                width * height
            }
            Shape::Circle(radius) => {
                3.141 * radius * radius
            }
        }
    }
}

let circle: Shape = Shape::Circle(5.0);
let area = circle.area();
println!("Area: {}", area);

\end{lstlisting}

We can use enums to solve some major problems, including nullability and error handling. \\

For example, to handle ``divide by zero'' errors, we can use the \codeword{Option} enum.
\begin{lstlisting}[frame = none]
// The `T` is a generic type, ignore for now.
enum Option<T> {
    None,
    Some(T),
}

fn divide(numerator: f32, denominator: f32) -> Option<f32> {
    // Check for div by zero
    if denominator == 0.0 {
        // We can't divide by zero, no float can be returned
        None
    } else {
        // denominator is nonzero, we can do the operation
        let quotient: f32 = numerator / denominator;

        // Can't just return `quotient` because it's `f32`
        Some(quotient)
    }
}

let quotient: Option<f32> = divide(10.0, 2.0);
println!("{:?}", quotient);

let zero_div: Option<f32> = divide(10.0, 0.0);
println!("{:?}", zero_div);

\end{lstlisting}

We can similarly account for null pointer names for greet(), which we coded in Lab 1. 

\begin{lstlisting}[frame = none]
fn greet(maybe_name: Option<&str>) {
    match maybe_name {
        Some(name) => println!("Hello, {}", name),
        None => println!("Who's there?"),
    }
}

greet(Some("William"));
greet(None);
\end{lstlisting}

The other big problem that can be handled by enums is error handling. 
To handle this sort of error, we can use the Result enum.
\begin{lstlisting}[frame = none]
enum Result<T, E> {
    // success
    Ok(T),
    // failure
    Err(E),
}
    
\end{lstlisting}

We can use the \codeword{?} operator to allow for error propagation:

\begin{lstlisting}[frame = none]
fn read_to_string(path: &str) -> Result<String, io::Error>
    
use std::{fs, io};

fn main() -> Result<(), io::Error> {
    let string: String = match fs::read_to_string("names.txt") {
        Ok(string) => string,
        Err(err) => return Err(err),
    };

    println!("{}", string);
    Ok(())
}
\end{lstlisting}

can be transformed to the following code:

\begin{lstlisting}[frame = none]
use std::{fs, io};

// the `Ok` value is `()`, the unit type.
fn main() -> Result<(), io::Error> {
    let string: String = fs::read_to_string("names.txt")?; // <-- here

    println!("{}", string);
    Ok(())
}

\end{lstlisting}

This can also work for functions that return option types, as follows:
\begin{lstlisting}[frame = none]
fn sum_first_two(vec: &[i32]) -> Option<i32> {
    // `.get(x)` might return `None` if the vec is empty.
    // If either of these `.get(x)`s fail, the function will
    // short circuit and return `None`.
    Some(vec.get(0)? + vec.get(1)?)
}
\end{lstlisting}

We can even write expressions with \codeword{?}, which is itself an expression! For example, we can do the following, which takes the result of a division and add 5 to it.
\begin{lstlisting}[frame = none][frame = none]
fn compute(a: f32, b: f32) -> Result<f32, String> {
    Ok(divide(n: a, d: b)? + 5.0)
}

\end{lstlisting}

An important caveat of the ``?'' syntax is that it only works with \codeword{Option} and \codeword{Result} types. \\

To summarize, we learned the following: \\

\begin{itemize}
    \item Rust enums are types that can be one of several variants which may contain different types.
    \item We use match statements to determine which variant an enum is.
    \item The problem of null pointers and references can be solved with enums like \codeword{Option<T>}
    \item Different languages have their own ways to mark a function as ``fallible", Rust has the \codeword{Result<T, E> enum}.
    \item The \codeword{?} operator can be used to propagate errors with minimal syntactic overhead.
    \item Enums are excellent for representing possible kinds of errors.
    \item The \codeword{?} operator can perform implicit conversion using the \codeword{From<T>} trait.
\end{itemize}