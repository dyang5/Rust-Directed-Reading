\section{Memory Management in Rust}

\subsection{Ownership}

\begin{definition}
Rust uses an \textbf{ownership model} to manage memory, with three fundamental rules.
\begin{enumerate}
\item Every value has an owner (variable/struct).
\item Every owner is unique.
\item When the owner goes out of scope, the value is dropped (freed).
\end{enumerate}
\end{definition}

\begin{lstlisting}[frame = none]
fn main() {
    // allocated on the heap
    let string = "Hello".to_string();
    // dropped
} // 'string' goes out of scope, so "Hello" freed


\end{lstlisting}


\begin{lstlisting}[frame = none]
fn say_string(string: String) {
    println!("{}", string);
}

fn main() {
    let string = "Hello".to_string();
    say_string(string);
    // running say_string(string) again would give an error, as the value
    // is used after the move.
} 


\end{lstlisting}

Ownership is checked at compile time. One approach would be to clone the string with string.clone and to then call the function on it. \\

Note that this ownership model \underline{does not} impact ``cheap types" like integers, booleans, and floats.

\subsection{References}
We can use `\&' for references. For example, we could pass a reference to an initial string and call say\_string on it just fine.

\begin{lstlisting}[frame = none]
fn say_string(string: &String) {
    println!("{}", string);
}

fn main() {
    let string = "Hello".to_string();
    say_string(&string);
    say_string(&string);
}
\end{lstlisting} 
    
Another example:
\begin{lstlisting}[frame = none]
fn main() { // this is okay, as a loses ownership to b.
    let b;
    {
        let a = "Hello".to_string();
        say_string(&a);
        b = a;
    }
    say_string(&b);
}
\end{lstlisting}

Let's now try to implement a counter.

\begin{lstlisting}[frame = none]
struct Counter {
    count: u32,
}

impl Counter {
    fn get_count(&self) -> u32 {
        self.count
    }

    fn increment(&mut self) {
        self.count += 1
    }
}

fn main() {
    let mut counter = Counter {count : 0}; 
    // mutable reference needs to have mutable ``var'' owner
}
\end{lstlisting}

% Add box around this
\textbf{You can take one mutable reference or as many immutable references as you want at a time (but not both).}

\begin{lstlisting}[frame = none]
let mut vec: vec![1, 2, 3];
let first = &vec[0];
vec.clear(); // mutable reference

println!("{}", first) // first is an immutable borrow, and we cannot have both at the same time
\end{lstlisting}

The main takeaways can be summarized as follows: \vspace{-0.1cm}\begin{itemize}
\item Every piece of data has a unique owner.
\item When that owner goes out of scope, the data is dropped.
\item Ownership can be transferred by moving or copying the data.
\item Data can also be borrowed via references to avoid unnecessary copying/moving.
\item References are guaranteed at compile time to always be valid.
\item Slices are references to contiguous chunks of memory.
\item You can't borrow something if it is already mutably borrowed, guaranteeing immutability.
\end{itemize}
